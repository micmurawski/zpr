Gra składa się z 2 komponentów z serwera i klienta które należy odzielnie skompilować


\begin{DoxyItemize}
\item Kompilacja klienta w folderze client\+\_\+window należy uruchamić kolejno komendy qmake -\/makefile i make -\/j4 (przy kompilacji na 4 rdzeniach)
\item Kompilacja serwera uruchamiamy komendę scons
\end{DoxyItemize}

\subsubsection*{Przykład}


\begin{DoxyItemize}
\item scons
\item cd client\+\_\+window
\item qmake -\/makefile
\item make -\/j4~\newline
 Wszystkie pliki zostaną skompilowane do katologu głównego
\end{DoxyItemize}

\section*{Uruchomienie Gry}

W celu uruchomienia gry należy urchomić serwer na następnie uruchomić klientów tak żeby jeden utworzył grę, a drugi do niej dołączył


\begin{DoxyItemize}
\item Uruchomienie serwera $>$./server -\/s \#host \#port
\item Urucomienie pliku ./server bez żadnych parametrów domyślnie uruchomi $>$./server -\/s 127.\+0.\+0.\+1 3002
\item Uruchomienie klienta gry z utworzeniem gry o zadanej nazwie na serwerze o zadanym ip i porcie $>$./client -\/c 127.\+0.\+0.\+1 3002 \#nazwa\+\_\+gracza \#nazwa\+\_\+gry
\item Uruchomienie klienta gry z dołączeniem do gry o zadanej nazwie na serwerze o zadanym ip i porcie $>$./client -\/j 127.\+0.\+0.\+1 3002 \#nazwa\+\_\+gracza \#nazwa\+\_\+gry
\item Wszystkie możlwości są opisane po uruchomieniu programu z parametrem -\/h
\end{DoxyItemize}

\subsubsection*{Przykład}


\begin{DoxyItemize}
\item ./server
\item ./client -\/c 127.\+0.\+0.\+1 3002 gracz1 gra1
\item ./client -\/j 127.\+0.\+0.\+1 3002 gracz2 gra1
\end{DoxyItemize}

\section*{Opis Gry}

Bedzię to turowa gra strategiczna odbywająca się na mapie w postaci grafu. Gracz będzie budował swoje statki wzamian za swoje zasoby i wysyłał je do węzłów grafu. W momencie kiedy dojdzie do spotkania staków przeciwnych graczy będzie następować wymiana ognia.


\begin{DoxyItemize}
\item Folder game\+\_\+engine zawiera Klasy \hyperlink{classGameServer}{Game\+Server},\hyperlink{classGameClient}{Game\+Client}, \hyperlink{classGameEngine}{Game\+Engine}.
\end{DoxyItemize}

\hyperlink{classGameServer}{Game\+Server} jest to klasa dopowiedzialna za serwer gry.

\hyperlink{classGameEngine}{Game\+Engine} odpowiada za przechowywanie stanu gry i generowanie nowego stanu na podstawie otrzymanych danych od graczy

\hyperlink{classGameClient}{Game\+Client} odpowiada za wastwę klienta gry


\begin{DoxyItemize}
\item Folder game\+\_\+object zawiera klasy realizujące obiekty gry
\end{DoxyItemize}

Klasa Game\+Object\+Factory działa jak fabryka skalowalna i generuje odpowiednie obiekty gry na podstawie otrzymanych stringów
\begin{DoxyItemize}
\item Folder game\+\_\+order przechowuje klasę Game\+Order która realizuje komendy wysyłane przez klientów gry
\item Folder game\+\_\+state przechowuje klasę Game\+State, która realizuje chwilowy stan gry, w póżniejszym czasie stan ten będzie renderowany na interfejsie gracza
\item Biblioteka to osbługi T\+CP została pobrana, zmodyfikowana o potrzebną funkcjonalność i lokalnie skompilowana jako biblioteka statyczna z repozytorium github \href{https://github.com/Cylix/tacopie}{\tt https\+://github.\+com/\+Cylix/tacopie}
\item Interfejs uzytkownika będzie napisany przy wykorzystaniu biblioteki Qt w wersji 5.\+5 \href{http://doc.qt.io/qt-5/linux-deployment.html}{\tt http\+://doc.\+qt.\+io/qt-\/5/linux-\/deployment.\+html}
\item Folder test\+\_\+suite przechowuje przykładowe testy jednostkowe kompilowane do pliku wykonywalnego test\+\_\+suite 
\end{DoxyItemize}